% ********** Rozdział 1 **********
\chapter{Opis założeń projektu}
\section{Cele projetu}
%\subsection{Tytuł pierwszego podpunktu}

Celem projektu jest stworzenie aplikacji konsolowej w języku C\# obsługującej podstawowe działania związane z zarządzaniem systemem odbioru paczek, działania które powinien wykonywać system zarządzania paczkomatem to nadawanie jak i odbieranie paczek, zarządzanie stanem przesyłki przez administratora oraz operacje ułatwiające pracę kurierów.

\section{Wymagania funkcjonale i niefunkcjonalne}


\noindent \textbf{Wymagania funkcjonalne}
\begin{itemize}
    \item Nadawanie oraz odbieranie paczek.
    \item Zmiana statusu zamówień.
    \item Dodawanie i usuwanie kurierów.
    \item Logowanie do systemu administratora z weryfikacją.
    \item Zablokowanie dostępu do paczek dla kuriera tylko do tych do których jest przypisany.
    \item Odbiór paczki tylko po podaniu odpowiedniego kodu odbioru.

\end{itemize}

\noindent \textbf{Wymagania niefunkcjonalne }
\begin{itemize}
    \item System powienien pozwalać na szybkie nadanie i odebranie paczki.    
    \item Proste i szybkie operacje użytkownika.
    \item Dostęp do działań administracyjnych tylko do pracowników z zezwoleniem.
    \item Bezpieczeństwo
    \item Zgodność danych z główną bazą danych
\end{itemize}
