
%spis rysunków
\addcontentsline{toc}{chapter}{Spis rysunków}
\listoffigures
\newpage

%spis tablic
\addcontentsline{toc}{chapter}{Spis tablic}
\listoftables
\newpage

 %streszczenie
 \addcontentsline{toc}{chapter}{Streszczenie}
 \noindent
 {\footnotesize{}\textbf{Wyższa Szkoła Informatyki i Zarządzania z siedzibą w Rzeszowie\\
 Kolegium Informatyki Stosowanej}
 \vspace{30pt}

 \begin{center}
 \textbf{Streszczenie projektu C\#}\\
 \temat
 \end{center}

 \vspace{30pt}
 \noindent
 \textbf{Autor: \autor
 \\Prowadzący: \promotor
 }
 \vspace{40pt}
  \\Aplikacja do zarządzania punktem odbioru paczek, połączona z bazą danych, spełniająca założenia projektowe.
 \vspace{80pt}

 \noindent
 \textbf{The University of Information Technology and Management in Rzeszow\\
 Faculty of Applied Information Technology}
 \vspace{30pt}

 \begin{center}
 \textbf{Project Summary\\}
Application managing parcel pick up point.
 \end{center}
 \vspace{30pt}
 \noindent
 \textbf{Author: \autor
 \\Supervisor: \promotor}
\vspace{40pt}
\\Application helps in managing parcel pick up point, it concludes safety and usability requirements.  
